
%
%  $Description: Project proposal for MushAI$ 
%
%  $Date: 2010 $
%

\documentclass[times, 10pt,twocolumn]{article} 
\usepackage{proposal}
\usepackage{times}
\usepackage{url}

%\documentstyle[times,art10,twocolumn,proposal]{article}

%------------------------------------------------------------------------- 
% take the % away on next line to produce the final camera-ready version 
\pagestyle{empty}

%------------------------------------------------------------------------- 
\begin{document}

\title{MushAI: Project proposal (group 8)}

\author{Hans Andersson\\
Program1\\mail\\
% For a paper whose authors are all at the same institution, 
% omit the following lines up until the closing ``}''.
% Additional authors and addresses can be added with ``\and'', 
% just like the second author.
\and
Erik Axelsson\\
Program2\\
SecondAuthor@institution2.com\\
\and
Erik Levin\\
MPSEN-1\\
levine@student.chalmers.se
\and
Max Ocklind\\
program4\\
mail
}

\maketitle
\thispagestyle{empty}

\begin{abstract}
Do we need an abstract for project proposal?
\end{abstract}



%------------------------------------------------------------------------- 
\Section{Summary}

Try to explain what you want to do and why. Particularly for those who have an area to research rather than a precise track: you must have a clearly defined concrete question, problem or task.

%------------------------------------------------------------------------- 
\Section{Key ideas}



%------------------------------------------------------------------------- 
\SubSection{What do you propose to do?}

It is important to define clearly the scope of the project.

%------------------------------------------------------------------------- 
\SubSection{What interesting and related things are outside the scope of your project?}



%------------------------------------------------------------------------- 
\SubSection{Why do you believe your approach will work?}



%------------------------------------------------------------------------ 
\SubSection{What will you accomplish?}

If you succeed, what will you accomplish? Why is it interesting and significant?


%------------------------------------------------------------------------- 
\section{Survey}
You should find published work or results that help you define the project, explain its significance, and demonstrate the value of various approaches to your task. If you cannot find work directly applicable to your project, try to find work on related problems, and adapt the results to your project.

\subsection{Literature}
One paper examining the nature of solvable games is \cite{games_solved}, in which several properties of games appear: game-theoretic value (first-player win, second-player win or draw), divergent/convergent, 4 categories of games... We will try to find out what of these properties apply to Traverse to anticipate whether the game is feasible to solve with current computational power.

\subsection{Tools and programs}

\subsection{Benchmarks}

%-------------------------------------------------------------------------
\section{Evaluation}

\begin{itemize}
	\item{How will you measure progress and results?}
	\item{What experiments will you perform to prove your hypothesis?}
	\item{Are there any standard data sets/benchmarks to evaluate the performance of your application?}
	\begin{itemize}
		\item{If YES, you should directly use those as a metric to evaluate the performance}
		\item{If NO, you should define a set of general metric to make the evaluation unbiased}
	\end{itemize}
	\item{For example you can collect some standard inputs from the real world scenario for your application, for which you know the actual results. Then you can compare the performance of your application against the actual results}
\end{itemize}

%-----------------------------------------------------------------------
\section{Initial results}
You should have a few small (or toy) programs as proof of concept, and to give reason to believe your approach will succeed.

%-----------------------------------------------------------------------
\section{Project management}
\begin{itemize}
	\item{What are the major tasks to accomplish?}
	\item{Who will do what?}
	\item{Try to define a tentative schedule showing when you will start and finish each task!}
\end{itemize}
However, please note that it is unwise to leave any task entirely to just one person. The others should at least follow it fairly closely, so that someone can take over the task in case of illness or absence.

%-----------------------------------------------------------------------
\section{Website}
The project, including a wiki, report drafts and code is published under GNU General Public License v3 and can be found at the following web adress:

\url{http://code.google.com/p/mushai/wiki/Start?tm=6}

\nocite{games_solved,course_book}
\bibliographystyle{latex8}
\bibliography{latex8}

\end{document}

